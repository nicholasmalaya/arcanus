\documentclass[11pt]{article}
\usepackage{graphicx,amsmath,amsfonts,amssymb,graphicx} 
\usepackage[varg]{txfonts}
\usepackage{enumerate}
\usepackage{hyperref}
\usepackage{listings}
\usepackage{color}
\urlstyle{tt}

\usepackage{geometry}
\geometry{%
  letterpaper,
  lmargin=2cm,
  rmargin=2cm,
  tmargin=2cm,
  bmargin=2cm,
  footskip=12pt,
  headheight=12pt}
  
\usepackage{lastpage}
\usepackage{fancyhdr}
%\pagestyle{fancy}
%\headheight 35pt

\def\squarebox#1{\hbox to #1{\hfill\vbox to #1{\vfill}}}
\def\qed{\hspace*{\fill}
        \vbox{\hrule\hbox{\vrule\squarebox{.667em}\vrule}\hrule}}
\newenvironment{solution}{\begin{trivlist}\item[]{\bf Solution:}}
                      {\textbf{//} \end{trivlist}}
\lstset{
	language=MATLAB,              % choose the language of the code ("language=Verilog" is popular as well)
   tabsize=3,							  % sets the size of the tabs in spaces (1 Tab is replaced with 3 spaces)
	basicstyle=\tiny,               % the size of the fonts that are used for the code
	numbers=left,                   % where to put the line-numbers
	numberstyle=\tiny,              % the size of the fonts that are used for the line-numbers
	stepnumber=2,                   % the step between two line-numbers. If it's 1 each line will be numbered
	numbersep=5pt,                  % how far the line-numbers are from the code
	%backgroundcolor=\color{mygrey}, % choose the background color. You must add \usepackage{color}
	%showspaces=false,              % show spaces adding particular underscores
	%showstringspaces=false,        % underline spaces within strings
	%showtabs=false,                % show tabs within strings adding particular underscores
	frame=single,	                 % adds a frame around the code
	tabsize=3,	                    % sets default tabsize to 2 spaces
	captionpos=b,                   % sets the caption-position to bottom
	breaklines=true,                % sets automatic line breaking
	breakatwhitespace=false,        % sets if automatic breaks should only happen at whitespace
	%escapeinside={\%*}{*)},        % if you want to add a comment within your code
	%commentstyle=\color{BrickRed}   % sets the comment style
}                    
\begin{document}

\title{\bf{CSE397: Assignment \#5}}
\author{Nicholas Malaya \\ Department of Mechanical Engineering \\
Institute for Computational Engineering and Sciences \\ University of
Texas at Austin} \date{} 
\maketitle
\newpage

\subsection*{Problem 1: An inverse problem for Burgers' Equation}

\begin{enumerate}
\item[(1)] Derive a weak form. Use integration-by-parts on the viscous
	   term to derive the weak form of Burgers' equation. 

\begin{solution}
Starting with Burgers' equation in strong form, 
\begin{align}
 u_t + u u_x - \nu u_{xx} = f, 
\end{align}
we multiply by a test function, $p(t,x)$ and integrate over time and space, 
\begin{align}
 \int_0^T \int_0^L (u_t p + u u_x p - \nu u_{xx} p - f p) dx dt = 0. 
\end{align}
Now, we integrate the viscous term by parts, to move a derivative of x
 onto the test function, 
\begin{align}
 \int_0^T \int_0^L (u_t p + u u_x p + \nu u_{x} p_x - f p) dx dt -
 \int_0^T \nu u_x p \bigg|_{x=0}^{x=T} dt = 0 
\end{align}

 
\end{solution}
\end{enumerate}

\newpage
\subsection*{Code}
%Here is the code for part c:
%\lstinputlisting{code/elliptic_sd_ip_adv_TV.m}
\end{document}