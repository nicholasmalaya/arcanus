\documentclass{article}
\usepackage{hyperref}
\usepackage{amsmath,amssymb}
\usepackage{graphicx}
\usepackage{caption}
\usepackage{subcaption}
\usepackage{color}
\usepackage[section]{placeins}
\usepackage{listings}


\title{Combustion Theory Final Take Home Exam} 
\author{Nicholas Malaya\\ Department of Mechanical Engineering \\
University of Texas at Austin}  
\date{}

\begin{document}
\maketitle
\newpage


\section*{Turbulent diffusion flames}

In this problem, I want you to assume that the system is turbulent and
that you know the turbulence mass diffusivity ($D_T$) is 10 times the
laminar value. Assume that the fluctuation squared of the mixture
fraction is equal to the gradient of the mean mixture fraction squared
multiplied by the characteristic diffusion length scale squared, i.e.
\begin{equation}
 \bar{z'z'} = \frac{1}{2} (\nabla \bar z)^2 \frac{D_T L_x}{u}
\end{equation}

\subsection*{a) Write down the solution for the mean mixture fraction
field with the turbulent diffusivity.}

We start with the steady state species equation, 
\begin{equation}
 \rho u \frac{\partial Y_i}{\partial x} = \rho D \frac{\partial^2
  Y_i}{\partial y^2} \pm \omega_i. 
\end{equation}
We note that, 
\begin{equation}
 \omega = \frac{\omega_i}{\nu_i W_i} = \frac{\omega_F}{\nu_F W_F} =
  \frac{\omega_O}{\nu_O W_O}. 
\end{equation}
This hints at a conserved scalar form of the species equation, where
with, 
\begin{equation}
 \beta  = \frac{\omega_F}{\nu_F W_F} - \frac{\omega_O}{\nu_O W_O}
\end{equation}
then our reaction is decoupled from the convection-diffusion of a
conserved scalar quantity. In particular, 
\begin{equation}
 \mathcal{L}(\beta)  = \mathcal{L}\left(\frac{\omega_F}{\nu_F W_F} -
				   \frac{\omega_O}{\nu_O W_O} \right)
 \Rightarrow \rho u \frac{\partial \beta_i}{\partial x} - \rho D \frac{\partial^2
  \beta_i}{\partial y^2} = 0. 
\end{equation}
Now, we construct z, 
\begin{equation}
 z = \frac{\beta -\beta_{O}}{\beta_F - \beta_{O}}
\end{equation}
Here, the boundary conditions are that $z=1$ for all $y>0$ and $x<0$
(e.g. the fuel reserve) and $z=0$ for $y<0$ and $x<0$ (e.g. the oxygen
reserve). 

Thus, we are solving, 
\begin{equation}
\rho u \frac{\partial z}{\partial x} - \rho D \frac{\partial^2
  z}{\partial y^2} = 0. 
\end{equation}
and, 
\begin{equation}
\rho u \frac{\partial \bar z}{\partial x} - \rho D_T \frac{\partial^2
  \bar z}{\partial y^2} = 0. 
\end{equation}
Where the first equation is from the laminar flow, and the latter case
is the favre-averaged mean field. We now need to discretize this
equation, in order to solve it numerically (It looks like it would be a
trainwreck to solve analytically!). We will completely wimp out, and
only use finite difference methods, 
\begin{align}
  \frac{\partial \bar z}{\partial x} &= \frac{\bar z_{i+1}-\bar z_{i}}{\Delta x} \\
  \label{first}
  \frac{\partial^2 \bar z}{\partial y^2} &= \frac{\bar z_{j+1}-2\bar z_{j}+\bar z_{j-1}}{\Delta y^2} 
\end{align}
The boundary conditions are $\bar z = 1$ $\forall x<0,y>0$, $\bar z = 0$
$\forall x<0,y<0$. I additionally imposed a Neuman (zero flux) boundary
condition on the top and bottom of the box, essentially forcing the
derivatives to zero at $\pm \infty$. As with the previous examination,  
we are now in a position to instantiate this on a computer using python
to solve for the $\bar z(x,y)$ field.  

As an interesting implementation detail, because the solution is only
first order in x, we only need one boundary condition for that
direction, as shown above. This is equivalent to an initial condition on
a first order (in time) ODE. Thus, we do not actually need to store our
entire x-domain in memory, but can simply solve for all y-values at our
particular x-coordinate, and then step forward in space and solve for
the next grid location. 

%
%
%
%
\subsection*{b) Plot the fluctuation and mean value of the mixture
fraction at 5 cm, 30 cm, and 50 cm.}

  \begin{figure}[!htb]
   \begin{center}
    \includegraphics[width = 12 cm]{figs/mean.pdf}
    \caption{The mixture fraction plotted as a function of y.}
    \label{mean}
   \end{center}
   \end{figure}

The method to arrive at the mean value mixture fraction was described
above. The value of the mixture fraction at several locations is plotted
in figure \ref{mean}. The flow at $x=0$ cm is a sharp interface between
the fuel and the oxidizer. As you move downstream, the fuel and oxidants 
mix, which will create a mixing layer that diffuses out and increases in
y-width as a function of distance downstream. 

The fluctuation $\bar{z'z'}$
must also be determined. Normally, this would require solving another
differential equation, and potentially using submodels for the scalar
dissipation rate as well. However, we were given a simplified expression
(model) for the variance, namely, 
\begin{equation}
 \bar{z'z'} = \frac{1}{2} (\nabla \bar z)^2 \frac{D_T L_x}{u}. 
\end{equation}
Variations in y will be much larger than variations in x, and this
expression  
can be simplified to be,
\begin{equation}
 \bar{z'z'} = \frac{1}{2} (\frac{\partial \bar z}{\partial y})^2 \frac{D_T L_x}{u}. 
 \label{fluc}
\end{equation}
This expression is a model for the variance of the mixture
fraction. Intuitively, this expression is reasonable,  as we expect the
variance (in some sense, our uncertainty) of the value of the mixture
fraction to be largest in regions with large gradients. In the mixing
layer, the gradient will be quite large near the layer ($y=0,x=0$) and expanding outward  at larger x. 

We discretize equation \ref{fluc} using the finite difference scheme
shown in equation \ref{first}, however, now the indicies are changed
from i to j, to reflect the different direction of the
derivative. However, upon running the code, I found that a forward
finite difference was very numerically noisy for the solution at $x=5$
cm. This was because the interface at this distance is still quite
sharp, and so squaring the derivative ``blew-up'' the noise. I therefore
switched a centered finite difference scheme, namely:
\begin{equation}
  \frac{\partial \bar z}{\partial y} = \frac{\bar z_{i+1}-\bar z_{i-1}}{2\Delta y}.
\end{equation}

  \begin{figure}[!htb]
   \begin{center}
    \includegraphics[width = 12 cm]{figs/fluc.pdf}
    \caption{The mixture fraction variance plotted as a function of y.}
    \label{fluc}
   \end{center}
  \end{figure}

The results are plotted in figure \ref{fluc}. The results of this figure
display that at anything but low values of x, the variance is nearly
zero. It is only near the sharp interface that the flow is
turbulent. Outside of just a few centimeters, the variance is very
nearly zero, implying essentialy laminar flow. As we move downstream,
the pdf opens up, as the mixing layer grows and entrains more fluid
around it. The peak also grows, implying that the turbulence grows to a
higher reynolds number, with larger fluctuations. 

%
%
%
%
\subsection*{c) Plot the PDF of the mixture fraction at two points, $y=0$ cm,
$x=30$ and at $y=15$ cm, $x=30$ cm.}

$P(z) =$ probability of z in $z+\Delta z$. If we knew the joint
probability density function we could calculate it directly. However, we
don't know P() (the PDF). Instead, we will use  what Peters calls the
``Presumed Shape PDF Approach''. We must pick a probability
distribution. We are limited to two parameter distributions for the
model to be closed, because we only possess  $\bar z$ and
$\bar{z'z'}$. Essentially, our choice is between a clipped Gaussian and
the Beta  function distribution. We will use the Beta distribution, as
it should have much more appropriate  limit behaviour for the mixing
layer. In particular, due to the effect of intermittency, we expect the
edges of the mixing layer to act like a delta function. This effect can
be captured by the Beta distribution. The distribution function pdf has
the form,  
\begin{equation}
P(\bar z) = \frac{\bar z^{\alpha-1}(1-\bar z)^{\beta-1}}{\Gamma(\alpha)\Gamma(\beta)}\Gamma(\alpha + \beta)
\end{equation}
We further define,
\begin{align}
\alpha &= \bar z \gamma \\
\beta  &= (1-\bar z) \gamma
\end{align}
Where the variable $\gamma$ is defined as, 
\begin{equation}
  \gamma = \frac{\bar z (1-\bar z)}{\bar{z'z'}^2} -1 \geq 0.
\end{equation}

For the two locations, we can already predict what the distributions
will look like. The first, at $y=0$, will be roughly gaussian, with a
mean centered around the expected value of the mixture fraction at that
location. Given that we expect the mixture fraction to be one half at that
location, the mean should be around that value as well. For the location
off-center, we expect a distribution that is skewed towards the side
with more fuel or oxidizer. This is the oxidizer side, so we expect the
mixture fraction pdf to have a mean less than zero. Furthermore, we note
from figure \ref{mean} that this is far away from the mixing layer. We
expect there to be essentially no mixing this far away, and therefore,
the distribution should be essentially completely oxidizer, e.g. a
distribution very sharply pushed up against zero. 

  \begin{figure}[!htb]
   \begin{center}
    \includegraphics[width = 12 cm]{figs/pdf.pdf}
    \caption{The distribution of the mixture fraction, using the assumed
    pdf approach (with an assumed $\beta$ distribution). The green line
    on the right is at $x=30,y=0$, and the blue distribution is at $x=30,y=15$.}
    \label{pdf}
   \end{center}
  \end{figure}

The results of this when instantiated numerically are shown in figure
\ref{pdf}. The results are precisely what we expected. The distribution
at $x=30,y=0$ is not exactly at 0.5, but it is nearly so (0.53). The
distribution is not skewed to either the fuel or oxidizer side. 
I expect that this is accurate. 
One might note
that the beta distribution, despite how sharp it is near zero, is still
giving non-trivial weight to less than zero mixture
fractions. Personally, I am skeptical this is not an over-estimate, and
it is likely a weakness of the assumed beta distribution for this edge
case. However, note that assuming a Gaussian pdf would break down even
more severely in this case. 

%
%
%
%
\subsection*{d) Plot the laminar and mean turbulent temperature
distributions at $x=30$ cm.}

We already know that the laminar profile will have a linear profile,
from the Burke-Schumann solution. This takes the form, 
\begin{equation}
a z + b = T(z). 
\label{eq}
\end{equation}
This is equivalent to saying that at each grid point, we expect the
temperature distribution to be a delta function around the temperature
predicted by the mixing fraction at that location. 

The turbulent system is more complicated. Now, we expect the non-zero
variance in mixing fraction concentration to impact the temperature. 
In order to account for the turbulent fluctuations, we need to integrate
over the probability density, e.g.
\begin{equation}
 \tilde T = \int P(\bar z) T(\bar z) dz
\end{equation}
Using equation \ref{eq}, this is equivalent to, 
\begin{equation}
 \tilde T = \int P(\bar z) (a \bar z + b) dz
\end{equation}
Essentially, this ``blurs'' the temperature profile by averaging each
location with its neighboring temperatures. 

  \begin{figure}[!htb]
   \begin{center}
    \includegraphics[width = 12 cm]{figs/temperature.pdf}
    \caption{The temperature profiles for the laminar (blue) and
    turbulent (green) simulations at $x=30$.}
    \label{temp}
   \end{center}
  \end{figure}

These results are plotted in figure \ref{temp}. These profiles are
qualitatively similar to what we expected. The peak is lower for the
turbulent profile, but the temperature peak is wider and the temperature
is elevated farther away from the mixing layer, due to the increased
mixing from the turbulence. I was surprised that the slope is
 not perfectly linear in the case of the laminar profile. 

\subsection*{e) Discuss the results.}

The entire formulation utilized chemical
equilibrium models to find the long time stable solution. The model
cannot predict extinction or ignition. This also assumes that the
chemical time scales are much smaller than the turbulence time scales,
e.g. that the Damkohler number is large. 

We are also ignoring instantaneous turbulence effects, on account of
using a Favre-Averaging scheme, instead of resolving the fine turbulence
scales. So these results are certainly not expected to be accurate
instanteously, but only represent mean quantities. Furthermore, they may
be far from realistic for some time, as the flow transitions to
turbulence and large eddies form downstream of the splitter.  

Finally, we either ``turned-on'' or ``turned-off'' the turbulence
(e.g. it was fully developed turbulence, or completely laminar). In
reality, the flow might be intermittent or laminar away from the mixing
layer at $y=0$ and turbulent near it. So a mixing of the models might be
more appropriate. Ideally, the flow would be fully turbulent near the
centerline, intermittent at the edges of the layer, and laminar
outside. This could be more accurately modeled using the ``composite PDF
approach for intermittency''. 

Given these assumptions, we would have to be careful using the results
of this model in a predictive context. Those caveats aside, the results
have the correct qualitative character for a flame, and so are certainly
much more useful than only expert opinion or conjecture. This is not to be too
negative, this is an extremely complex system we are simulating, and I
am able to generate all the plot used in this report in a few seconds on
my computer. 


\newpage
All the work contained in this report was entirely my own. 

Thank you for the class! 
\vspace{1in}
\newline
References:

``Turbulent Combustion'', Norbert Peters

``Combustion Physics'', Chung K. Law


\subsection*{Code}
I wrote these routines entire from scratch, using python 2.X. The only
libraries necessary to run these routines should be Numpy, SciPy and
Matplotlib.  
\lstinputlisting{flow.py}

\end{document}
