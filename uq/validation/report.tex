\documentclass{article}
\usepackage{hyperref}
\usepackage{amsmath,amssymb}
\usepackage{graphicx}
\usepackage{caption}
\usepackage{subcaption}
\usepackage{color}
\usepackage[section]{placeins}
\renewcommand{\thesubsection}{\thesection.\alph{subsection}}
\usepackage{listings}

\title{Validation and Uncertainty Quantification Proposal for a
Solar-driven Vortex Apparatus} 
\author{Nicholas Malaya\\ Department of Mechanical Engineering \\
University of Texas at Austin}  
\date{}

\begin{document}
\maketitle
\newpage

\section{Introduction}
Much of the solar energy incident on the Earth's surface is absorbed
into the ground, which in turn heats the air layer above the surface.
This buoyant air layer contains considerable gravitational potential
energy. The basic idea behind this approach is to convert the 
potential energy in this buoyant air layer to kinetic energy in an
anchored vortex, and to use that kinetic energy to drive a
vertical-axis turbine coupled with an electric generator in order to
produce electrical power. Computational Fluid Dynamics (CFD) is used to
simulate this Solar-Driven Vortex (SoV). These computer simulations are
intended to discover the optimal system configuration for a range of
scenarios and system sizes. In order for these simulations to be
generally useful, they must first be validated against existing
experimental data and high fidelity simulations. These models
will then explore regimes and scales where no experimental measurements
presently exist. Characterizing the uncertainty of predictions resulting
from extrapolation is a critical component in enabling reliable
assessments of field performance of the SoV, as it will guide the
commercialization strategy of the product. 

This report provides a ``validation roadmap'' detailing the process by
which an analysis of the uncertainties inherent to this
problem may be characterized.  It will begin with a discussion of the
physics scenario and the mathematical model, as well as detailing the
reliability of each submodel and the systems inputs. We will then
discuss the validation of these models against existing experimental
data and high fidelity simulations. Finally, we will formulate a Bayesian
analysis for a sub-problem, to serve as a representative example of a
probabilistic analysis applied to this project.  


%
% end
%
\newpage
This work is supported by the Department of Energy [Advanced Research
Projects Agency-Energy] under Award Number [DE-FOA-0000670].   

\end{document}
