\documentclass{article}
\usepackage{amsmath,amssymb}

\title{\bf{Blending Function}}
\author{Nicholas Malaya \\ Institute for Computational Engineering and Sciences \\ University of Texas at Austin} \date{}

\begin{document}
\maketitle

We are constructing a blending function so that we can smoothly patch together 
the values of the penalty functions across a hybrid vane configuration, 
instead of discontinuously forcing. 

In order to accomplish this, we desire a blending function that is symmetric 
around a value, varying continuously from zero (no forcing) to one (full forcing). 
The blending function B(z) is chosen to be: 
\begin{equation}
  B(z) = \frac{\text{tanh}(\beta z -\gamma) + \alpha}{2}
\end{equation}

The value of $\beta$ dictates the thickness of the blending function. 
We choose a relatively sharp $\beta$ of nine, which makes the variation from 
zero to one occur over approximately one fourth of a meter. We use an identical 
magnitude, but opposite in sign, for the "other side" of the penalty function, which then 
varies from one to zero. 

$\alpha$ must be set to one in order to guarantee 
that the function starts at zero and ends at one. 

Finally, $\gamma$ determines the horizontal shift. This is set to the location where 
the vanes intercept. 

\end{document}
\documentclass{article}
\usepackage{amsmath,amssymb}

\title{\bf{Blending Function}}
\author{Nicholas Malaya \\ Institute for Computational Engineering and Sciences \\ University of Texas at Austin} \date{}

\begin{document}
\maketitle

We are constructing a blending function so that we can smoothly patch together 
the values of the penalty functions across a hybrid vane configuration, 
instead of discontinuously forcing. 

In order to accomplish this, we desire a blending function that is symmetric 
around a value, varying continuously from zero (no forcing) to one (full forcing). 
The blending function B(z) is chosen to be: 
\begin{equation}
  B(z) = \frac{\text{tanh}(\beta z -\gamma) + \alpha}{2}
\end{equation}

The value of $\beta$ dictates the thickness of the blending function. 
We choose a relatively sharp $\beta$ of nine, which makes the variation from 
zero to one occur over approximately one fourth of a meter. We use an identical 
magnitude, but opposite in sign, for the "other side" of the penalty function, which then 
varies from one to zero. 

$\alpha$ must be set to one in order to guarantee 
that the function starts at zero and ends at one. 

Finally, $\gamma$ determines the horizontal shift. This is set to the location where 
the vanes intercept. 

\end{document}
