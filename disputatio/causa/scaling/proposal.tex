\documentclass[english]{article}
\usepackage[usenames,dvipsnames,svgnames,table]{xcolor}
\usepackage{babel,blindtext}
\usepackage{graphicx}
\usepackage{caption}
\usepackage{subcaption}
\usepackage{tabularx}
\usepackage{amsmath}
\usepackage{mathrsfs}
\usepackage{setspace}
\usepackage{enumerate}
\usepackage{todonotes}
\usepackage{listings}
\usepackage{paralist}
\usepackage{natbib}
\usepackage{fullpage}
\usepackage[T1]{fontenc}
\usepackage{booktabs}
\usepackage{pgfplotstable}
\usepackage{makecell}
\usepackage{comment}
\usepackage{gensymb}

\usepackage[colorlinks=true,
            linkcolor=blue,
            urlcolor=blue,
            citecolor=blue,
            final,
            hypertexnames=false]{hyperref}

%
%
%
\title{\bf{SoV Scaling Study}}
\author{Nicholas Malaya, Robert D. Moser \\ Institute for Computational
Engineering and Sciences \\ University of Texas at Austin} \date{} 

\begin{document}
\maketitle

\section*{System Configurations}

\begin{table}
\begin{centering}
  \begin{tabular}{ | l || c | r | r | r | r |}
    \hline     
    Case & $r_{ib}$ & $r_{it}$ & $r_{o}$ & $z_{b}$ & $z_{t}$ \\ \hline \hline
    1m   & 0.16     & 0.33     & 0.50    & 0.066   & 0.55 \\ \hline
    3m   & 0.50     & 1.00     & 1.50    & 0.200   & 1.65 \\ \hline
    5m   & 0.75     & 1.50     & 2.50    & 0.333   & 2.75 \\
    \hline 
  \end{tabular}
  \caption{The three different vane configurations. $r_{ib}$ is the inner radius of the bottom tier, 
    $r_{it}$ the inner radius of the top tier, and $r_{o}$ the outer radius (shared by both tiers). 
    $z_{b}$ and $z_{t}$ are the heights of the top and bottom tiers, respectively. }\label{fig:scaling_table}
\end{centering}
\end{table}
%

\begin{figure}[!htb]
  \begin{center}
    \includegraphics[width = 12 cm]{figs/vane_scaling}
    \caption{An image showing the three different vane configurations. The three-meter vanes (mid-sized set) 
      have a slight plotting artifact at the top outer corner, but this does not exist in the simulations. }
    \label{vane_scaling}
  \end{center}
\end{figure}


This document details investigations into the variation of the Solar
Vortex (SoV) physical size and energy as a function of diameter of the
apparatus. We consider three fully developed, averaged runs for system
configurations that are identical excepting having been scaled to larger
sizes. We will use system apparatus that have outer vane diameters of
one, three and five meters. The design is two tiers, with a longer
curved vane on the bottom tier providing a smoother and slower variation
in angle as a function of radius. Each configuration has an outer angle
of $0\degree$, and varying along a smooth curve depending on the square
root of the radius inward to a maximum angle of $75\degree$. The
``baseline'' configuration was the three meter design, which was
optimized over the coarse of a parameter sweep across inner and outer
radius, heights, angles, etc. These different run definitions are
completely defined in Table \ref{fig:scaling_table}. The results shown
here are all temporally averaged, and are not sensitive to changes in
the domain size. 
%
% image of vane angle?
%
Figure \ref{vane_scaling} plots a vertical slice through the three vane
sizes, to give the reader a sense of the scale between each configuration.

\begin{figure}[!htb]
\minipage{0.32\textwidth}
  \includegraphics[width=\linewidth]{figs/1m_temp_iso}
  \caption*{1m Apparatus}\label{fig:1m_scaling}
\endminipage\hfill
\minipage{0.32\textwidth}
  \includegraphics[width=\linewidth]{figs/3m_temp_iso}
  \caption*{3m Apparatus}\label{fig:3m_scaling}
\endminipage\hfill
\minipage{0.32\textwidth}%
  \includegraphics[width=\linewidth]{figs/5m_temp_iso}
  \caption*{5m Apparatus}\label{fig:5m_scaling}
\endminipage
\caption{3d Isocontour Rendering of the Thermal Plume across three different cases.}
\label{fig:iso}
\end{figure}


\begin{figure}[!htb]
  \begin{center}
    \includegraphics[width = 12 cm]{figs/temp_iso}
    \caption{Thermal Isocontours from the three different cases superimposed 
      on top of each other to show the growing extent of the thermal column at larger vane sizes. }
    \label{fig:scaling_slice}
  \end{center}
\end{figure}

\begin{figure}[!htb]
  \begin{center}
    \includegraphics[width = 12 cm]{figs/scaling_regression}
    \caption{The thermal column thickness plotted against the total system diameter sizes. A linear regression is also shown against the data.}
    \label{fig:scaling_reg}
  \end{center}
\end{figure}

\section*{Thermal Plume Scaling}

Isocontours of the temperature field (contoured at 317 Kelvin) are 
shown in figure \ref{fig:iso}. 
It is clear that as the vanes grow in size, the physical size of the
thermal plume also increases. This is shown slightly more clearly in
Figure \ref{fig:scaling_slice}. Here, the two larger cases isocountours
are sliced and transparent so that one can see an approximate location
of the thermal column thickness for each of the three cases. These
thicknesses have been extracted and plotted as a function of outer
system diameter in Figure \ref{fig:scaling_reg}, where it is clear that
the trend is generally linear. It appears then that the thickness of the
thermal plume is scaling roughly proportionally with the system outer
diameter.  

%
% probably need to plot the increase in diameter as a function of diameter 
%


\begin{figure}[!htb]
\minipage{0.32\textwidth}
\includegraphics[width=\linewidth]{figs/w_scaled_1m}
\caption*{1m}\label{fig:1m_vz}
\endminipage\hfill
\minipage{0.32\textwidth}
\includegraphics[width=\linewidth]{figs/w_scaled_3m}
\caption*{3m}\label{fig:3m_vz}
\endminipage\hfill
\minipage{0.32\textwidth}%
\includegraphics[width=\linewidth]{figs/w_scaled_5m}
  \caption*{5m}\label{fig:5m_vz}
\endminipage
\caption{Vertical Velocity}
\label{fig:vz_scaling}
\end{figure}
 
Figure \ref{fig:vz_scaling} shows the vertical velocity for each of the
three cases. In each case the aspect ratio between domain extentand
apparatus size are maintained roughly constant, so as to give the
appearance of ``scaling'' each apparatus. In this way, it is more clear
here than in the thermal column case that there exists a qualitatively
similar structure between each of the three cases. The velocity plume
size occupies roughly the same fraction of the volume inside the vanes.  

%
%
%
\section*{Predictions for System Scaling}

%We can use our model to make some very rough predictions for system sizes to attain various power outputs. 

A crucial question now is how do we expect the kinetic energy to scale with larger system sizes. In other words, 
how do we determine the coefficient, $\alpha$, that dictates the power
at which the diameter is raised that determines the kinetic energy: 
\begin{equation}
k \propto D^\alpha. 
\end{equation}

Preliminary analysis has shown that the exponent should be to the fourth
power. We will investigate what this scaling might be by calculating it
data from the field simulations.  

\begin{figure}[!htb]
  \begin{center}
    \includegraphics[width = 12 cm]{figs/ke_exponent}
    \caption{The kinetic energy fluxes for the three configurations on a
   log-log plot}
    \label{fig:ke_exponent}
  \end{center}
\end{figure}

Figure \ref{fig:ke_exponent} is a first attempt at measuring the
exponent of $\alpha$. We plot the kinetic energy for our available
runs on a log-log plot. We calculate the kinetic energy flux as, 
\begin{equation}
k = \frac{1}{2} \int v_z (v_z^2 + v_{\theta}^2) dA. 
\label{ke_flux}
\end{equation}

Measuring the slope on an equal aspect-ratio log-log plot provides an estimate of the exponent. 
Measuring from our present data, the kinetic energy is scaling with the diameter raised to the power of 2.428. 
While non-integer scalings are not impossible, and not unprecidented in
turbulence, there are no obvious characteristic scales that would result
in this power-law exponent, and more pen and paper analysis is
necessary. Regardless, the result is sufficiently close to 2.5 as to be
intruiging, however, as that implies an anomalous scaling coefficient of
$\frac{5}{2}$.  

% The exponent would likely be an integer power, although this is not necessarily so. 

\begin{figure}[!htb]
  \begin{center}
    \includegraphics[width = 12 cm]{figs/ke_regression}
    \caption{The kinetic energy fluxes for the three configurations extrapolated out to larger configuration sizes.}
    \label{fig:ke_regression}
  \end{center}
\end{figure}

An interesting question is to ask if these predictions are consistent with the available experimental results. 
Sinclair (1973) has experimentally measured profiles, which can be integrated using equation \ref{ke_flux} to 
provide an estimate of the energy flux. This is measured to be $\approx 2.4$ kW. Examining the thermal 
profiles from Sinclair, the thermal plume thickness is found to be roughly 4.6 meters in diameter. This is 
7.6 times larger than the thermal plume thickness measured in the five-meter configuration. If we assume 
linear scaling in vane configurations, that would require a 38 meter outer diameter configuration to 
generate the same energy flux as the Sinclair field dust-devils. Figure \ref{fig:ke_regression} shows 
several powers of extrapolations from the present data sets to the larger diameter sizes. Our projections 
are roughly consistent with the Sinclair data for a scaling coefficient between two and four, but this extrapolation
could be unreliable. We are extrapolating far outside of the data, at scales and energies considerably 
exceeding the present data. As a result, while these data are not inconsistent with those reported by Sinclair, 
more data (at larger system diameters) is needed to improve upon this prediction. 

\end{document}
