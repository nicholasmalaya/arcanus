\documentclass[english]{article}
\usepackage[usenames,dvipsnames,svgnames,table]{xcolor}
\usepackage{babel,blindtext}
\usepackage{graphicx}
\usepackage{caption}
\usepackage{subcaption}
\usepackage{tabularx}
\usepackage{amsmath}
\usepackage{mathrsfs}
\usepackage{setspace}
\usepackage{enumerate}
\usepackage{todonotes}
\usepackage{listings}
\usepackage{paralist}
\usepackage{natbib}
\usepackage{fullpage}
\usepackage[T1]{fontenc}
\usepackage{booktabs}
\usepackage{pgfplotstable}
\usepackage{makecell}
\usepackage{comment}
\usepackage{gensymb}

\usepackage[colorlinks=true,
            linkcolor=blue,
            urlcolor=blue,
            citecolor=blue,
            final,
            hypertexnames=false]{hyperref}

%
%
%
\title{\bf{SoV Scaling Study}}
\author{Nicholas Malaya, Robert D. Moser \\ Institute for Computational Engineering and Sciences \\ University of Texas at Austin} \date{}

\begin{document}
\maketitle

Here, we investigate the variation of the Solar Vortex (SoV) as a function of diameter of the apparatus. We consider 
three fully developed, averaged runs for system configurations that are identical excepting having been scaled to larger sizes. We will use 
system apparatus that have diameters of one, three and five meters. The design is two tiers, with a longer curved vane on the bottom tier providing a 
smoother and slower variation in angle as a function of radius. Each configuration has an outer angle of $0\degree$, and varying along a smooth curve 
depending on the square root of the radius inward to a maximum angle of $75\degree$. The ``baseline'' configuration was the three meter design, 
which was optimized over the coarse of a parameter sweep across inner and outer radius, heights, angles, etc. 

% image of vane angle?
\begin{tabular}{ | l || c | r | r | r | r |}
  \hline     
  Case & $r_{ib}$ & $r_{it}$ & $r_{o}$ & $z_{tb}$ & $z_{tt}$ \\ \hline \hline
  1m & 2 & 3 & 3 & 3 & 3 \\ \hline
  3m & 2 & 3 & 3 & 3 & 3 \\ \hline
  5m & 2 & 3 & 3 & 3 & 3 \\
  \hline 
\end{tabular}


\begin{figure}[!htb]
  \begin{center}
    \includegraphics[width = 12 cm]{figs/vane_scaling}
    \caption{An image showing the three different vane configurations. The three-meter vanes (mid-sized set) 
      have a slight plotting artifact at the top outer corner, but this does not exist in the simulations. }
    \label{vane_scaling}
  \end{center}
\end{figure}

Figure \ref{vane_scaling} plots a vertical slice through the three vane sizes, to give the reader a sense of the 
scale between each configuration.

\begin{figure}[!htb]
\minipage{0.32\textwidth}
  \includegraphics[width=\linewidth]{figs/1m_temp_iso}
  \caption{1m Apparatus}\label{fig:1m_scaling}
\endminipage\hfill
\minipage{0.32\textwidth}
  \includegraphics[width=\linewidth]{figs/3m_temp_iso}
  \caption{3m Apparatus}\label{fig:3m_scaling}
\endminipage\hfill
\minipage{0.32\textwidth}%
  \includegraphics[width=\linewidth]{figs/5m_temp_iso}
  \caption{5m Apparatus}\label{fig:5m_scaling}
\endminipage
\end{figure}

An isocontour of the temperature field (contoured at 317 Kelvin) are shown in figures \ref{fig:1m_scaling}-\ref{fig:5m_scaling}. 
It is clear that as the vanes grow in size, the physical size also increases. 

Should certainly mention that these results do not appear to be sensitive to differing domain sizes. 

%
% just a figure
%

  %% \begin{figure}[!htb]
  %%   \begin{center}
  %%    \includegraphics[width = 12 cm]{figs/1m_temp_iso}
  %%    \caption{1m }
  %%    \label{lab}
  %%   \end{center}
  %% \end{figure}


\end{document}
