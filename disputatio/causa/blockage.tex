\documentclass{article}
\usepackage{amsmath,amssymb}
\usepackage{bm}

\title{\bf{Actuator Disk: Blade Blockage Modification}}
\author{Nicholas Malaya \\ Institute for Computational Engineering and Sciences \\ University of Texas at Austin} \date{}

\begin{document}
\maketitle

The length of each blade blocking vertical flow in the actuator disk is, 
\begin{equation}
  Ch * \text{Cos}(\beta(r)) = Ch_x(r). 
\end{equation}
The total length in an annular region that is blocking is therefore,
\begin{equation}
  N * Ch_x(r) = l_B(r),
\end{equation}
Where N is the number of blades. 
Then, the ratio of an annular length that is impeded by blades is, 
\begin{eqnarray}
 B(r) =& \frac{2\pi r- l_B(r)}{2 \pi r}\\
 B(r) =& 1- \frac{l_B(r)}{2 \pi r}. 
\end{eqnarray}
Note that a ``floor'' function is needed to ensure the blockage ratio
does not go below zero. 

From a 1D control volume analysis for the region, with the continuity
equation, 
\begin{eqnarray}
 \rho V_z' A' =& \rho V_z A,\\
 V_z' =& V_z \frac{A}{A'}, \\
 V_z' =& \frac{V_z}{B(r)}.
\end{eqnarray}
Note that this implies that $V_z' \rightarrow \infty$ as $A' \rightarrow
0$. This is as expected, as the blockage becomes more severe, the flow
would need to move at greater speed to go through it. 

Our flow angle with respect to turbine velocity is therefore modified
from,

\begin{equation}
 \theta_f = \text{tan}^{-1}(\frac{U_{\text{up}}}{U_{\text{fwd}}})
\end{equation}
to,
\begin{equation}
 \theta_f = \text{tan}^{-1}(\frac{U_{\text{up}}}{B(r) U_{\text{fwd}}}). 
\end{equation}
As $B(r) \rightarrow 0$, $\theta_f \rightarrow \pi$, e.g. 90 degrees. In
this way the velocity vector will be completely aligned with drag. 

Furthermore, 
\begin{equation}
 F = \frac{1}{2}\frac{c \rho U_p^2}{A}(C_L \cdot n_{\text{lift}} + C_d \cdot n_{\text{drag}})
\end{equation}
Is modified so that, 
\begin{equation}
 \bar U_p^{2} = \left(\frac{U_p}{B(r)}\right)^2
\end{equation}

\end{document}
