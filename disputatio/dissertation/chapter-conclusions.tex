\label{sec:conclusions}

\section{Summary of the Present Work}

% CLEMENS: "Is this true?" underlining "approximating...on".
Turbulent boundary layers approximating those found on the NASA Orion
Multi-Purpose Crew Vehicle (MPCV) thermal protection system during atmospheric
reentry from the International Space Station have been studied by direct
numerical simulation, with the ultimate goal of reducing aerothermodynamic heating prediction
uncertainty.
%
Simulations were performed using a new, well-verified, openly available
Fourier/B-spline pseudospectral code called Suzerain equipped with a recent,
``slow growth'' spatiotemporal homogenization approximation developed by
\citet{Topalian2014Spatiotemporal}.
%
A first study aimed to reduce turbulence-driven heating prediction uncertainty
by providing high-quality data suitable for calibrating Reynolds-averaged Navier--Stokes turbulence
models to address the atypical boundary layer characteristics found in such reentry problems.
%
The unique boundary layer data includes strong favorable pressure gradients,
cold isothermal wall conditions, and wall transpiration effects and has
well-quantified uncertainties so that it may best inform turbulence models.
%
A second study aimed to reduce transition-driven uncertainty by determining where
on the thermal protection system surface the boundary layer could sustain
turbulence.
%
This study informs where fully laminar and where fully turbulent assumptions are
appropriate in the reentry scenario without incurring the uncertainties
associated with transition modeling.

%%%%%%%%%%%
% CHAPTER 6
%%%%%%%%%%%

In the first study, the two data sets generated and investigated were a
$\Mach\approx{}0.9$ and a $\Mach\approx{}1.15$ spatiotemporally homogenized
boundary layer with $\Reynolds[\theta]{}\approx{}382$ and $\Reynolds[\theta]{}\approx{}531$,
respectively.  Boundary layer edge-to-wall temperature ratios were approximately
4.15 and wall blowing velocities, measured in plus units, were in the neighborhood of
\num{8e-3}.  The favorable pressure gradients, achieved by supplying a
stationary inviscid flow profile to the homogenization approximation, had
acceleration parameters~\citep{Launder1964Laminarization} of about \num{4e-6}
and Pohlhausen parameters between 25 and 42.  Skin frictions coefficients around \num{6e-3}
and Nusselt numbers under 22 were observed.  Due to the considerable
thermodynamic property gradients, the subsonic simulation had an unexpectedly
small displacement thickness while the supersonic simulation exhibited negative
displacement effects.  As a consequence, the Clauser
parameter~\citep{Clauser1954Turbulent} was found misleading for characterizing
these pressure gradients.  Objective uncertainty
estimates~\citep{Oliver2014Estimating} for the data found coefficients of
variation of less than \num{8e-3} for density, velocity, temperature, and
viscosity inside the boundary layer edge and of roughly 10\% for the specific
turbulent kinetic energy for statistical ensembles gathered for 6.4--6.9 eddy
turnover times.  Semi-local scaling~\citep{Huang1995Compressible} collapsed all
profiles investigated.  The near-wall vorticity fluctuations show qualitatively
different profiles than those from the incompressible~\citep{Spalart1988Direct} or
compressible literature~\citep{Guarini2000Direct}.  The turbulent Prandtl number
was above 0.8 inside the boundary layer edge.  Root-mean-squared property
fluctuations matched expectations for isothermal wall
conditions~\citep{Coleman1995Numerical} but the supersonic results show evidence
of minor problems in the numerical formulation related to the spatiotemporal
homogenization when $\Mach>1$.  Favre-averaged equation budgets were reported and
show the regions in which the homogenization approximation directly impacts the mean
flow.  In particular, the direct slow growth influence on the total energy and
turbulent kinetic energy equations is small enough that
such homogenized flows can serve as convenient model problems for calibration
of models to be used in spatially evolving boundary layers.

%%%%%%%%%%%
% CHAPTER 7
%%%%%%%%%%%

%A second study aimed to reduce transition-driven uncertainty by detecting where
%on the thermal protection system surface the boundary layer could sustain
%turbulence.
%%
%This study informs where fully laminar and where fully turbulent assumptions are
%appropriate in the reentry scenario without incurring the uncertainties
%associated with transition modeling.

In the second study, local boundary layer conditions were extracted from a
laminar flow solution over the Orion MPCV thermal protection system during peak
reentry heating which included shock effects, aerothermochemistry, curvature, and
ablation.  That information, as a function of leeward distance from the
stagnation point along the MPCV symmetry plane, was approximated by
$\Reynolds[\theta]{}$, $\Mach[e]{}$, $p_{e,\xi}^{\ast}=\frac{\delta}{\rho_e
u_e^2}\frac{\partial\!p_e}{\partial\!\xi}$, $v_w^{+} = v_w / u_\tau$, and
$T_e/T_w$ along with perfect gas assumptions.
%
Homogenized turbulent boundary layers were initialized at those local conditions
and evolved until either stationarity, implying the conditions could sustain
turbulence, or relaminarization, implying the conditions could not.
%
A computationally convenient periodic domain, which fluctuations cannot exit,
served as a surrogate for a perturbation-rich flight environment.
%
Fully turbulent fields relaminarized subject to conditions
4.134~m and 3.199~m leeward of the stagnation point.
%
At those two locations, $\Reynolds[\theta]{}\approx{}225$, $\Mach[e]{}>0.9$, and
$T_e/T_w\approx{}4.1$ all approach maxima over the heat shield while
$p_{e,\xi}^\ast$ and $v_w^{+}$ become small (see
Figures~\ref{fig:cevisslam_summary1} and~\ref{fig:cevisslam_summary_fpg}).
%
These results suggest that nowhere on the MPCV thermal protection system can
sustain turbulence in this reentry scenario.
%
However, different and somewhat pathological initial conditions unexpectedly
produced a long-lived, fluctuating field at leeward position 2.299~m.  No
evidence of turbulence-sustaining behavior appeared at leeward position 1.389~m.
Accordingly, it was predicted that locations more than 1.389~m leeward of the
stagnation point can sustain turbulence in this scenario.
%
Relaminarization for the \citet{Topalian2014Spatiotemporal} homogenized boundary
layers showed similar early and late time behavior as that described by
\citet{Cal2008Similarity} for spatially evolving flows.

\section{Recommendations for Future Work}

Regarding the first study, more investigation into the basic character of the
spatiotemporally homogenized boundary layers produced by the
\citet{Topalian2014Spatiotemporal} technique is warranted.  With a better
understanding of the behavior of the approach on simpler cases, it would
be possible to disentangle the combined influence of homogenization, strong
favorable pressure gradients, cold walls, and wall transpiration.  Future
simulations might begin from the $\Reynolds[\theta]{}\approx{}382$ simulation
presented here and incrementally eliminate complicating features to produce a
sequence of problems approaching more canonical, better understood flows.
%
Several symptoms of minor problems with the present numerical formulation of the
homogenization for $\Mach>1$ were raised, including in one dimension for laminar
solutions, suggesting that additional analysis of the homogenization is
worthwhile.  Though found usable here, it may be the case that either
straightforward Giles-like nonreflecting boundary
conditions~\citep{Giles1988Nonreflecting,Giles1990Nonreflecting} or the chosen
isothermal wall enforcement scheme or both are inappropriate.  These symptoms
may also indicate a malady in the model itself.
%
Applying the homogenization approach to either the low Mach number,
variable density or incompressible limits of the Navier--Stokes equations
would be worthwhile.  Doing so also may provide insight regarding
the issues observed from the present numerical formulation.

Regarding the second study, it would be interesting to recompute the Orion MPCV
thermal protection system flow for peak heating during International Space
Station reentry using our estimate of the edge of the turbulence-sustaining
region.  From this mixed laminar/turbulent solution the energy flux to the
ablator could be compared against the fully laminar or fully turbulent results.
That computation would also provide a
prediction against which flight data from the upcoming NASA Exploration Flight
Test-1~\citep{SpaceCom20140317} might be compared.
%
Obtaining a more precise location for the edge of the turbulence-sustaining
region will require methodology improvements.
%
In particular, a turbulence-model-based procedure to find the code inputs
yielding the desired $\delta_{99}$, $\Reynolds[\theta]{}$, $\Mach[e]{}$,
$p_{e,\xi}^{\ast}$, $v_w^{+}$, and $T_e/T_w$ would permit continuing the present
relaminarization study based on the fluctuation-sustaining conditions found at 2.299~m.
%
The present ad hoc approach for discovering input parameters, though it produced
interesting and useful results, was operationally unsatisfying and its effective
application below 2.299~m would be difficult.
%
The constant wall blowing $v_w^{+}$ designed to emulate steady-state
outgassing from the ablator in future studies might better be replaced
by a controller-based mechanism to adjust the wall blowing velocity to
achieve nondimensional heat fluxes $B_q$~\citep{Bradshaw1977Compressible}
extracted from the original laminar flow solution for the reentry
scenario.
%
With such improvements in place, investigating higher-speed reentry trajectories
for the Orion MPCV might simplify characterizing the edge of a
turbulence-sustaining region given fully turbulent initial
conditions and also be more fruitful from the perspective of improving basic
understanding of the physics in these scenarios.
%
Finally, the validity of the present methodology for detecting
turbulence-sustaining regions might be investigated by comparing its predictions
against experimental transition data gathered from a wind tunnel facility that
has been configured to be exceptionally noisy.
%
%%%EVIL%EVIL%EVIL%%%
\enlargethispage{1.0em}
%%%EVIL%EVIL%EVIL%%%
