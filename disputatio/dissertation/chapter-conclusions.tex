\label{sec:conclusions}

%
% be sure to mention that this thesis is a success of CFD/CAD
% computational optimization and computational design
%

%
% bob mentioned today that one should considered feasibility in the 
% context of just technological scaling, no cost. 
%

\section{Summary of the Present Work}

%
% be sure to hedge bet as technological feasibility is limited to just
% tech, not finance
%

%
% weave every story together--
%
%     1) results are not promising
%     2) be sure to talk about new models contributions and possible
%     applications 
%

Observed velocities indicates that naturally occurring dust devils
contain a considerable amount of energy, from the gravitational
potential energy contained in air near the hot surface of the Earth and
ambient winds. While rudimentary proof-of-concept devices have been
developed that show that artificial dust devils may be created and
anchored in place by turning vanes, no previous studies have attempted
to methodically explore how to engineer these control surfaces to
intensify the produced vortex, and in doing so, yield a stronger and
more powerful dust devil. 

This thesis has developed and validated computational models that have
explored a broad configuration space of turning vanes so as to estimate
how much power might be produced by one of these synthetic dust devils,
and in doing so, provide an assessment of the technological feasibility
of the entire synthetic columnar vortex concept as a means of generating
usable energy. 

Such a system had never been simulated previously. Developing
simulations capable of modeling the SoV apparatus and for the scenarios
of interest therefore required the development of new models and
software capabilities. This required the development and implementation
of particular mathematical models for the diffusivities in the ambient
conditions, and the formulation for the SoV vanes, cone and
turbine. The latter required the development of a novel representation
of the SoV system geometry that is sufficiently flexible to permit
cost-effective iteration in designs. These virtual vanes were
further extended to include a separation model and the drag due to skin
friction along the surface. In addition, a modification to the
actuator-disk model to account for the observed impact of the turbine
blade solidity was developed. 

The ultimate simulation capability was a complicated coupled system that
accounted for buoyancy effects, ambient winds and a wide variety of
turning vane and turbine geometries. These simulations were validated
across a range of conditions and system configurations and were largely
consistent with available experimental data and
observations. Furthermore, a steady state simulation was validated
against unsteady simulations, which permitted computationally
inexpensive explorations of the design space. 


%In doing so, we
%have discussed the physics that influence dust devils formation, our
% We summarized the
%numerical discretizations used, the software stack and the calibration,
%verification and validation of these components. The purpose of these
%sections was to communicate two major points. 
%The first is that an
%accurate, verified and validated simulation capability has been
%developed that can quickly investigate a wide variety of system and
%scenario settings at a modest computational cost. 

% keep?
%The second point is
%that we have developed heuristics that permit optimization of any
%baseline SoV configuration to a local maximum of energy production, as
%measured by kinetic energy flux through the top of the SoV vanes.  

These capabilities supported the principle objective of this work, which
is to explore a large space of possible system configurations and
geometries to discover mechanisms that intensify the vortex velocities
and the power extracted by the turbine. These simulations indicated that
the synthetic flow generated by the SoV does indeed have a dust
devil-like character. The flow is buoyancy driven, with a coherent
thermal plume corresponding to a region of intense vertical and
azimuthal velocity. These flows are consistent with the Rankene
vortex model, which in turn has generally been found to be an accurate
representation of the velocities in the naturally occurring phenomenon. 

A contribution of this work is in indicating that role of
ambient winds are more substantial than previously indicated. The wind
makes a significant contribution of kinetic energy to the flow, but it
also changes the structure of the vortex. The expectation that the more
powerful dust devils are driven by substantial winds also implies that
the structure of dust devils may be more asymmetric than previously believed. 


%Coupled with the scaling analysis presented in
%Chapter~\ref{sec:physics}, we are now able to predict the conditions (if
%any) under which the SoV apparatus will be technologically competitive
%with other sources of renewable energy.   
%This has also permitted investigating the physics of the apparatus, to
%assess how closely the synthetic dust devils mimic the natural
%variety. 

%
% probably dont want this...
%
\section{System Feasibility Assessment}

The results of the simulations indicate that over tens of square meters,
several kilowatts of energy can be produced. These results are not
promising with regards to the competitiveness of the power generated by
the device.\todo{compare to existing technologies} 

Furthermore, at this time no experimental validation accompanies the
computational results, and it is conceivable that the apparatus will not
perform as well as predicted.  

It is also worth noting that the synthetic dust devils are fragile, and
the kinetic energy flux in the plume is highly sensitive to ambient
conditions and the complex interplay between the ambient winds and the
thermal buoyancy effects. In some cases, small perturbations to the
system parameters such as the turbine blade number or the turning vane
angles would result in orders of magnitude weaker vorticies. 
This indicates that any energy generation system that uses dust
devils may not be a reliable form of energy generation. This it
may indicate that the naturally occurring dust devils, while pervasive,
are not robust, and that individual whirlwinds are easily dissipated
when interfered with. 

However, it must also be emphasized that feasibility is focused on
technical viability, namely energy produced by the apparatus, and does
not include an economic assessment. The SoV is almost certainly cheaper
to fabricate and install than competing renewable technologies such as
photovoltaics or wind turbines. Thus, while simulations indicate that
the SoV lags these technologies in terms of power production, it may be
more competitive on a dollar per watt basis. 

%In other words, it is currently believed that
%that the SoV does produce usable energy,  
%but the design required to do could be prohibitively expensive, and
%therefore not economically competitive with existing technologies. 


% is this too hokey?
%In essence, our calculus is, 
%\begin{equation}
% = \frac{\text{Energy}}{\text{meter}^2} * \text{ efficiency (\%) } *
%  \text{meter}^2 
%\end{equation}
%Where the peak energy per meter squared is detailed in Section~\ref{},
%and the efficiency estimated in Section\ref{sec:peak_estimate}. 

%A pertinent question is how does this compare to existing technologies,
%when evaluated from a similar standpoint?

One outstanding question is that the synthetic dust devils generated by
the SoV possess substantially reduced velocities versus the naturally
occurring phenomenon. It is not clear why the velocities are lower than
expected. The scaling of the velocity of these objects with
respect to the diameter of the apparatus is not known, and it is
possible that the larger diameters of the naturally occurring dust
devils also corresponds to intensified velocities. Alternatively,
the solidity or blockage of the device in the vanes and turbine was
found to be a significant limit to the power produced. As the natural
variety do not have control surfaces that also serve to block the flow,
these vortices may have larger mass flow rates and kinetic energy
fluxes.  
 

\section{Conclusions and Future Work}

Discuss how this is a feedback loop\todo{might need adjustable vanes to
'build up' vortex}

This thesis is nevertheless a success with regards to the objectives of
development of a simulation capability. The optimization and 

\todo{add note that mentions how the formulation of an optimization
heuristic is also a worthy contribution}

fundamental fluid structure? the failure of several models during the
optimization work performed here 


Only positive venue for this are in 
completely different scenarios (waste industrial heat)
with much more exotic energy

physics hints that multiple tiers are necessary to make a sort of
continuous set of entrainment work

might mention that more expensive models are def possible\todo{multiple
turbines} -- still pretty speculative since usually diminishing


what continuing mysteries\todo{inconsistent results for 2016 field test}



testbed for cyclonic phenomena

% \subsection{Additional Investigations}

% % 
% % control for inter unit spacing
% % 
%In addition to the system configuration, it would be interest to
%consider the effect of local conditions on SoV
%performance. Characterizing the impact of variations in ambient
%conditions on the SoV\todo{finish me}

% will guide the commercialization strategy of the
% product, by determining optimal install locations across the country. It
% is therefore desirable to have models that are capable of accounting for
% variation in field conditions, such as solar input, cross-winds and
% topography. Furthermore, it is expected that large ``farms'' of SoVs
% (akin to the wind and solar farms for wind turbines and photovoltaics,
% respectively) may be used by commercial or utility-scale energy
% generation. For this to be effective,  the inter-unit spacing must also
% be optimized, as a single SoV collects from a large area. These
% computations will guide commercialization planning, where
% decision-makers will need to assess optimum unit size, spacing, and
% geographic location for utility-scale deployment.   
