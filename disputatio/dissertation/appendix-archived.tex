\label{sec:archiving}

%Do you want vane forcing functions defined here?

The entirety of data used in this document have been captured and
are on the tape archival system
Ranch\footnote{npm7@ranch:/home2/00000/npm7/sov\_huge\_backup} 
at the Texas Advanced Computing Center\footnote{% 
    \url{http://www.tacc.utexas.edu/}
}
(TACC).  These complete archives will be made available on request. 

The files are stored in a format identical to that of the SVN archive
located at,
\url{https://svn.ices.utexas.edu/repos/pecos/solar_vortex/}. The
organization of the repository bears some discussion. The root level,
contains the folders: {\it documents}, {\it grids},{\it input}, {\it
postproc}, and {\it single\_shot\_input}. {\it documents} contains
quarterly reports and model documentation in LaTeX and MSFT word format.  
{\it grids} contains the raw gridgen files used to generate meshes for
the gridded vanes. {\it postproc} contains the files used to perform
temporal averaging, as well as paraview and python scripts used to
visualize the fields and generate images of the simulations. {\it
single\_shot\_input} is a deprecated set of input files from the
earliest investigations into the SoV. These input files represent an
older format where all the definitions and file settings were contained
in a single input file. Due to the volume and complexity of input
required for these simulations, these older files are cumbersome and
extremely difficult to read. Finally, {\it input} is the directory that
contains the input files as well as the output of the simulations (on
Ranch, not on SVN). 

The {\it input} directory is broken into four directories. The first,
{\it field}, contains all the physical investigations for the SoV Field
tests with the virtual vanes, typically steady, but some unsteady
virtual vane investigations are also contained here. {\it gridded} are
directories that contain the input files and simulation output from the
gridded runs, and {\it laboratory} contains the table-top laboratory
runs. Finally, {\it opt} contains the optimization runs where runs where
rapidly iterated with perturbed system parameters. These were entirely
``Steady'' virtual vane cases. 

All these directories then have a common structure. They have a {\it
common} directory that contains all the sub-input files, and then a
unique problem folder that details the unique run, as well as the output
of this file. For instance, the a problem folder might be entitled,
``field\_2016\_august\_3m'' for the 2016 August field test conducted
with a three meter per second wind velocity. Inside each problem
directory, there are two files, an ``initial.in'' and a ``gold.in''. The
initial file starts a run, even if steady, typically with enhanced (and
likely un-physical) viscosity, to help the solver converge. Subsequent
runs are restarted from this base state but with the viscosity model
detailed in Chapter~\ref{sec:mathmodel}. No results from this initial
solve are quoted in this document. In some cases for complicated
geometries, multiple initial steps were required, in which the viscosity
was stepped down from the high initial state to the model derived
values. 

In addition to the input files, each of these directories contains two
directories, {\it gold} and {\it output}. After each run, all of the
output files are moved into {\it output} and

This is handled automatically, by custom bash scripts attached to the
slurm scheduler. These scripts are available in the {\it common}
directory. 

\begin{table}
\centering
\caption[Execution details captured from each production batch job]{%
  Software and hardware execution details captured from every production batch
  job as human-readable text files.  Files named like \texttt{*.dat} provide time
  measured relative to a wall clock, the simulation physics, and time step
  number.\label{tbl:executiondetails}
}
\begin{small}
\begin{tabular}{p{0.20\textwidth}|p{0.70\textwidth}}
Filename & Contents \\ \hline \hline
\texttt{bc.dat}       & Trace of conserved state behavior at boundaries \\
\texttt{binary}       & Absolute path to the compiled Suzerain binary \\
\texttt{cpuinfo}      & \texttt{/proc/cpuinfo} from MPI rank zero \\
\texttt{dependencies} & Runtime-resolved shared library dependencies \\
\texttt{environment}  & Environment variables in effect at runtime \\
\texttt{kernel}       & \texttt{/proc/kernel} from MPI rank zero \\
\texttt{log.dat}      & Complete execution log according to Suzerain \\
\texttt{meminfo}      & \texttt{/proc/meminfo} from MPI rank zero \\
\texttt{output}       & Complete execution log according to the batch system \\
\texttt{qoi.dat}      & Trace of scalar quantities of interest like $\Reynolds[\theta]{}$ \\
\texttt{state.dat}    & Trace of mean and fluctuating conserved state \\
\texttt{version}      & Suzerain version information from the compiled binary
\end{tabular}
\end{small}
\end{table}

The second type of data archived contains instantaneous physics.  This data
consists of complete instantaneous field snapshots taken periodically during each
simulation run along with a variety of scenario parameters and descriptive grid
statistics.  

\begin{table}
\centering
\caption[Instantaneous fields and other details comprising a restart file]{%
  A small subset of the details comprising a Suzerain restart file.
  HDF5 comments in the file provide operational context.  For example,
  information on field storage ordering is provided in the comments of
  \texttt{/rho}, \texttt{/rho\_E}, etc.\label{tbl:restartfile}
}
\begin{small}
\begin{tabular}{p{0.29\textwidth}|p{0.65\textwidth}}
HDF5 Dataset & Contents \\ \hline \hline
\texttt{alpha                 } & Ratio of bulk to dynamic viscosity \\
\texttt{t                     } & Simulation physical time 
\end{tabular}
\end{small}
\end{table}

This data is cumbersome and not easily imported into common software
like GNU~Octave, \textsc{Matlab}\textsuperscript{\textregistered},
\textit{Mathematica}\textsuperscript{\textregistered}, or Python in a
single command. Rather, paraview provides the best means to visualize
and explore these datasets, and was the main post-processing software
used in this thesis. 

The github hashes of the latest (and believed to be backward compatible)
GRINS and libMesh versions used in this document were, 
\begin{verbatim}
  GRINS Version: 5373d0fc001ea98c715638851e4e3b0e7f96cc95                                                  
  libMesh Version: cd139a10cef2cf603f85f64a11c10d6bbe4d6780
\end{verbatim}
%
While built from master in the development branch, these should
correspond closely to versions v0.7.0 in GRINS and libMesh v1.0.0.  

%Finally, the raw .tex of this document and the accompanying figures was
%is publicly available at,
%\url{https://github.com/nicholasmalaya/arcanus/disputatio/dissertation}.  