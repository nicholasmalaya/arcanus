\section{Model}

\begin{itemize}
\item unstable stratified boundary layers (raleigh number estimate)
\item justify incompressible N-S
\item justification of far-field eddy-viscosity model (M-O)
\item modeling eddy-viscosity in device 
\item vane and turbine representation via penalty function // immersed boundary method
\item cone representation
\end{itemize}

%remember that \st{} is strikethrough
%
% should this all be math modeling?
%

Our aim is to simulate the formation of synthetic dust devils in the
field. This requires a model of the ambient conditions for a
representative case, such as Arizona, where experimental data is
available from test have been performed. Furthermore, for this to be
more generally useful in the prediction of flows in a variety of
conditions, we need a model generally applicable to any flow near the
surface of the earth.  

This section details an analysis of surface fluid mechanics, and
develops a theory of turbulence in a thermally stratified medium. As we
are utilizing a RANS model with spatially variable diffusivity, we are
particularly interested formulating a model for these quantities. 

We are interested in the operation of the apparatus during the day. 
At these times, the atmospheric surface layer has the following character. 
Incident radiation from the sun largely does not interact with the
air, which is nearly transparent. Instead, this radiation is absorbed by
the ground, which causes a temperature rise. This results in a thermal
gradient between the hot ground and the cooler air. The warm ground
conducts heat to the air, causing an expansion and lowering the density
of the air. This reduced density air near the surface is driven upwards
by the force of buoyancy.  

For sufficiently large temperature gradients, these motions are
unstable, and as the warm air is driven upwards the flow will transition
to turbulence. For the typical use case we consider, namely Arizona in summer, 
Rayleigh numbers are typically between $10^9 - 10^11$, and are therefore 
well in excess of the criterion for transition to a turbulent regime. The 
flow is that of an unstably statified fluid. 

\subsection{Viscosity Model}

We utilize the celebrated similarity model of Monin and Yaglom\cite{} as
a guide to the present development, which we outline below. This work is
extension of the mixing-length model of Prandtl, where the concepts of
gradient diffusion and mixing length were generalized to thermally
stratified flow.  

We begin by using dimensional analysis, and noting that the dynamics of
any mean quantity ($\bar f$) in a thermally stratified medium only depend on,

\begin{equation}
\bar f = f(z_0,\frac{g}{T_0},\rho_0,\nu,k,u^*,q)
\end{equation}

We expect that aside from very near the surface, the diffusivities $\nu$
and $k$ will be small compared to their turbulent counter-parts, $\nu_T$
and $K_T$. Likewise, if we define $z-z_0$ as an ``effective roughness
height'' or displacement distance, we can reasonably neglect $z_0$ from these
considerations. While the roughness height can be large (for instance in
a cornfield, where the roughness height could reasonably be several
meters), for our present study the expectation is that this roughness
height will be on the order of centimeters\cite{}. 

This leaves only five parameters: the distance from the ground, z; the
buoyancy coefficient, $\frac{g}{T_0}$; the density of the fluid,
$\rho_0$; a velocity scale, $u^*$; and the heat flux from the ground,
$q$. 
%
% add refence to dynamical and physical meteorology 
% 
These quantities depend on
four dimensions: length, time, temperature and mass. As a result,
Buckingham Pi theorem implies that only one dimension-less group can be
formed\cite{}.%munson 

This group is chosen to be,
\begin{equation}
 \xi = \frac{z}{L}.
\end{equation}
Here, $L$ is the famous, ``Monin-Obukhov'' length,
\begin{equation}
 L = -\frac{{u^*}^3}{\kappa \frac{g}{T_0} \frac{q}{c_p \rho_0}}
\end{equation}
where $\kappa$ is the (dimensionless) Von-Karman constant. Notice that
our interest lies in regimes where $L<0$, as $q<0$ (e.g. heat from the
ground into the fluid), which corresponds to the unstable stratification 
we expect during a sunny day. 

The physical meaning of L, the Monin-Obukhov length, bears some discussion. 
The numerator is proportional to the kinetic energy generated by the ambient 
shear of the fluid. The denominator is the buoyant production of kinetic energy. 
As a result, this length scale can be interpreted as the physical location where the
production of buoyant kinetic energy is approximately equal to the energy generated 
by wind shear. In other words, when the magnitude of L is large, the flow is dominated
by shear effects, and the impact of buoyancy is diminishingly small. Conversely, a small 
magnitude of L implies that buoyant effects largely dominate the kinetic energy production.

We are now in a position to state that the mean quantity has a
functional representation to the effect,
\begin{equation}
 \bar f = C \phi(\xi)
\end{equation}
with C a multiplicative constant with units of $\bar f$, and $\phi$ is a
function only of $\xi$. 

\section*{Asymptotic Behaviour of the Function, $\varphi$}

The asymptotic behavior of the $\varphi_T$ and $\varphi_u$ at
large and small values of $\xi$ provides guidance on more general character of the
functions. We are interested in the case where $L<0$, which corresponds to 
heat flux from the ground into the air.  

The case where $\xi \to -\infty $ implies $\frac{z}{L} \to
-\infty $ and $z>>L$. This is most readily interpreted as the instance
where $u^* \to 0$, e.g. the case with no wind. For this case, the
function $\varphi_T$ must hold no dependence on $u^*$, and will approach
a constant value. A glance at
equation \ref{eq:tz} shows that it holds a $1/4$ scaling in $u^*$. This
a result of $T^* \thicksim \frac{1}{u^*}$, and $L \thicksim
{u^*}^3$. $\xi$ holds a cubic dependence in $u^*$ through the dependence
on the M-O length, $L$. Therefore, the overall function will not depend
on $u^*$ only when the function $\varphi$ scales to the
$-\frac{4}{3}$ power. 

Regardless, with possession of the vertical temperature gradient, and
using equation \ref{eqn:eddy_kt},  
we are now in a position to describe the eddy thermal diffusivity as, 
\begin{equation}
 K_T = \frac{1}{C_T} \left( \frac{q}{c_p \rho_0} \frac{g}{T_0}
		     \right)^\frac{1}{3} z^{\frac{4}{3}}  \text{ 
for } z \gg L. 
\end{equation}

So long as the Prandtl number remains constant in space\cite{}, then
% todo: provide discussion as to why this is not an unreasonable expectation
identical arguments as to the asymptotic behaviour at large $\xi$ provide
the analogous result for the eddy viscosity's variation with respect to
distance from the ground,  
\begin{equation}
 \nu_T = \frac{1}{C_{\nu_T}} \left( \frac{q}{c_p \rho_0} \frac{g}{T_0}
			     \right)^\frac{1}{3} z^{\frac{4}{3}}  \text{ 
for } z \gg L. 
\end{equation}

Monin and Obukhov values for these constants of $C_{\nu_T} = 0.95$ and
$C_{T} = 3.6$.  

These functions have been found to be broadly applicable, accurate and 
are easily instantiated in software. Several known (and often well
characterized) shortcomings of the Monin Obukhov similarity theory
exist. These include: 

\begin{itemize}
 \item surfaces with large spatial variations in roughness heights
 \item outside of the surface layer (several hundred meters), the
       coriolis effect is no longer negligible, and must be accounted for
 \item even in ``ideal'' situations, the theory has seldom been found to
      be accurate to more than 10\% (kansas measurements)\cite{}
\item the theory's predictions are well known to be sensitive to choice
      of universal function for $L>0$.
\end{itemize}


\subsection{equations}

then it discusses equations
then it discusses the viscosity model
