
%\section{Calibration and Validation}
\section{Validation}
\label{sec:validation}

% VALIDATION TODO
%
% show major effort
% table of results
% show hierarchy
% precisely define cases 
%
%

%
% meshed vanes are 24x more expensive
%

The previous sections briefly outlined the physical phenomenon under
consideration, the mathematical models proposed to simulate it,
and the numerical solution of these models for a variety of system 
configurations and scenarios. Before these simulations can be used 
as a tool to evaluate proposed system designs,
it is necessary to validate that the model accurately represents reality.  
This section contains a discussion of the validation of the computational models
against existing experimental data and high fidelity simulations.

A challenge in this project is the scarcity of experimental data. Only
two or three cases of experimental measurements are available. These
measurements, for reasons detailed in the next subsection, are not
sufficient to provide confidence in the output of simulations across a
wide variety  of scenarios. Thus, a high fidelity model using explicitly
gridded vanes with enforced no-slip velocity boundary conditions on the
surface of the vane has been developed. These ``gridded'' runs have been
evaluated against the experimental data, which they match quite closely
(as described in the next subsection). However, as detailed in
\ref{subsec:vane}, explicitly meshing the vanes would be far too
prohibitively expensive to permit a rapid exploration of a variety of
system configurations. Instead, this high fidelity model is used as a
means to generate additional truth data to permit validation of lower
fidelity models, such as the virtual vanes. Likewise, the results of the
unsteady virtual vane simulations can be used as validation data for a
further reduced order, steady navier-stokes model. This hierarchy of
validation is depicted in \ref{fig:val_hier}, with data sources that
generate more reliable data at the top, and models that are less
reliable, but also less computationally expensive at the bottom. In
terms of expense, the reduced order model (ROM) generates a solution in
approximately two minutes, versus twelve hours for the unsteady virtual
vane model. The gridded vanes require another factor of ten in
computational time, and many more man-hours hours of work to generate
the mesh. Therefore, it is unrealistic to perform parameter sweeps or
system configuration investigations with the gridded vanes. Instead, the
ROM is used, with promising results re-evaluated in the unsteady VV
models, and so on up the hierarchy if more confidence is necessary.

%
% https://www.draw.io/
%
 \begin{figure}[!htb]
   \begin{center}
    \includegraphics[width = 8 cm]{figs/validation_hierarchy}
    \caption{This figures depicts the validation hierarchy between
    different sources of data generation. The experimental measurements
    lies at the top, where the data is expected to be the most reliable,
    but simultaneously the most rare. Moving down the table leads to
    simulated data sources that are less reliable but increasingly
    cheaper in time to generate. At the bottom is the reduced order
    model (ROM).} 
    \label{fig:val_hier}
   \end{center}
 \end{figure}

Three kinds of validation data are available. Thes are comparisons
between data generated in the heated laboratory and the gridded and
(steady) virtual vanes (``thermal-only''), comparisons between
experiments in a wind tunnel against gridded and virtual vanes
(``Wind-only''), and comparisons between the virtual vanes and the field
tests (``Field'') conducted in Arizona. The existing available data from
these cases as well as the gridded vanes created to mimic them are are
summarized in table \ref{tab:val_data}. Due to the formulation ease,
every case shown has been simulated using the virtual vanes.  

\large
\begin{table}[h]
\centering
\label{my-label}
\begin{tabular}{l|l|l|l|}
           & Wind-Only                   & Thermal-Only               & Field  \\
  \hline 
Experiment & Straight Vanes $60^{\circ}$ & Straight Vanes $60^{\circ}$ & June 2014   \\
           &                           & Straight Vanes $30^{\circ}$   & August 2014 \\
           &                           & Hybrid                        & August 2015 \\
  \hline 
Gridded    & Straight Vanes $60^{\circ}$ & Straight Vanes $60^{\circ}$ & \\
           & Straight Vanes $30^{\circ}$ & Straight Vanes $30^{\circ}$ & \\
  \hline 
\end{tabular}
  \caption{Presently available data from the laboratory experiments 
    (cold wind and thermal-only) as well as the field test as well as
 the gridded vanes. } 
  \label{tab:val_data}
\end{table}
%
%
%
% http://www.tablesgenerator.com/
%
%

%
% experimental challenges
%
\normalsize
\subsection{Thermal-Only Validation}
This section provides examples of the validation performed on the
richest data set, the measurements in the laboratory. All of the
thermal-only the data was generated in a laboratory setting at Georgia
Tech. This general system configuration is depicted in Figure
\ref{fig:lab_image}. These data were taken using particle image
velocimetery (PIV), and the errors in in measurement and sampling are
not known. In addition, only velocity measurements are
available. Several potentially important quantities of interest, such as
the pressure field or the temperature, have not been measured. 

%
% http://convertonlinefree.com/ConvertImageEN.aspx
%
 \begin{figure}[!htb]
   \begin{center}
    \includegraphics[width = 12 cm]{figs/Optimized-lab_setup}
    \caption{The single tier straight vane laboratory configuration. The
    apparatus is shown with a turbine, but it was removed for data
    gathering.}
    \label{fig:lab_image}
   \end{center}
 \end{figure}

While no sensitivity analysis has been performed, it is likely that the
largest uncertainty in the laboratory simulation is a result of the
ventilation. The heated plate at the bottom of the apparatus
generates enough heat to cause a significant increase in room
temperature (30+ Kelvin), which greatly impacts the SoV
performance, as the ground to air thermal gradient drives the
vortex. The laboratory is cooled to maintain
temperature, in particular, two inlet HVAC ducts into the room. 
%While efforts have been made to characterize the level of ventilation being
%used, these numbers come with non-trivial uncertainties attached. 
One vent runs continuously at 288 Kelvin with a flow rate estimated 
to be 1 $m^3$/s.
%(4-6 m/s with an approximate area of 0.2 $m^2$)
The other vent is active only if the room temperature exceeds 301 Kelvin, 
with a flow rate also estimated at 1 $m^3$/s.
Finally, the air leaves through the cracks around the laboratory doors and 
exhaust vents. Preliminary results indicated that an inflow rate of 1
$m^3$/s, the lower bound of the possible inflow rates results in
excessive heating of the room, while inflow conditions at the maximum
inflow rate of 2 $m^3$/s result in a simulated room that is too cold,
compared to the laboratory.  

Our simulated vortices are sensitive to ambient room temperature and thus 
the inflow rate. It is likely that the laboratory is run where one of
the vents is operating intermittently. 
To mimic these conditions in our simulations, Dirichlet boundary conditions 
on parts of the sides of the computational domain are used to
establish a constant inflow of cool air at the rates 
proscribed by our collaborators. Over the remainder of the side walls, 
adiabatic thermal boundary conditions are are used. 

The most signficant boundary condition disparity is that flow leaves the
domain through the top boundary, instead of out of the sides of the
room. Preliminary results suggested that the SoV phenomenon  was not
sensistive to these boundary condition details. The important element is
the  global energy balance in the room. The flow rate into the room is
adjusted to  1.3 $m^3$/s for the validation results discussed here.  

%% \begin{figure}[!htb]
%%   \begin{center}
%%    \includegraphics[width = 12 cm]{figs/hybrid_profile}
%%    \caption{Azimuthal and vertical velocity profiles as a function of
%%    radius. The simulation and experimental data broadly agree, with
%%    the simulation also exhibiting the characteristic ``twin-peak''
%%    structure of the hybrid vanes in the azimuthal velocity. }
%%    \label{fig:lab}
%%   \end{center}
%% \end{figure}

\begin{figure}[]

 \begin{subfigure}{.5\textwidth}
  \centering
  \includegraphics[width =0.9\textwidth]{figs/sim_vs_exp_30_vt}
  \caption{Azimuthal velocity as a function of radius for the thermal-only cases.
    The gold line is the virtual vane case, blue the experiment, and red the gridded.}
 \end{subfigure}%
 \begin{subfigure}{.55\textwidth}
  \centering
  \includegraphics[width =0.9\textwidth]{figs/sim_vs_exp_30_vz}%
  \caption{Vertical velocity as a function of radius for the thermal-only cases. 
The gold line is the virtual vane case, blue the experiment, and red the gridded. } 
 \end{subfigure}%
  \label{fig:val_lab}  
\end{figure}

Figure \ref{fig:val_lab}\todo{fix this image} is a direct comparisons
between laboratory 
measurements for a simple single tier configuration ($30^{\circ}$
straight vanes) and nominally identical simulations with the gridded and
virtual vanes. The simulations and experiment broadly agree. The
simulation correctly reproduces the peak structure in the azimuthal
velocity observed for this configuration in the experiment. The gridded
vanes very closely represent the peaks radial location, while the
virtual vanes over-predict the location. The radial location of peak
vertical velocity also closely agrees with experiment.

Similar validation comparisons have been made between several other
configurations with similar levels of agreement,  notably the
$60^{\circ}$ single tier straight vane case, and the two tier hybrid
vane.  Finally. estimates of the energy fluxes between the field
configuration and our simulation results agreed within 15\%. These
validation studies have provided a modest level of confidence that our
simulations accurately reproduce the dynamics observed in real vane
configurations.\todo{finish discussion here} 

Qualitative comparisons have been made between the our simulation
results with a mean velocity and with wind tunnel pictures. These images
showed roughly consistent structure between simulation and experiment. 

%
% validation story is incomplete
% 
% you have done: 
%
% 1) comparions between laboratory + gridded + virtual 
% 
% 2) comparisons between gridded + virtual in laboratory
%    + field configurations, thermal only and wind
%
% 3) comparisons of virtual vanes to field observations 
%    quantitative + qualitative
%
%
% You need to discuss what needs to be validated -- the
% comparison to gridded vanes is a useful validation tool 
%
