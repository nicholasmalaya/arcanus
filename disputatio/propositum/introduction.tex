\section{Introduction}

Renewable energy is critical to our environmental, economic, and
national security. Demand for energy is on the rise, as is our national
reliance on fossil fuel-based power plants for the bulk of our
electricity generation. There is a critical need for safe, clean, and
cost-effective alternatives to coal, such as wind, solar, hydroelectric,
and geothermal power\cite{arpa-e}. These technologies would reduce carbon dioxide
emissions and help position the U.S. as a leader in the global renewable
energy industry. 
%
% proposal
%
This proposal details a research plan perform a numerical investigation
for the design and optimization of a novel device for renewable, clean
energy generation. 

Much of the solar energy incident on the Earth's surface is absorbed
into the ground, which in turn heats the air layer above the surface.
This buoyant air layer contains considerable gravitational potential
energy. The basic idea behind this engineering approach is to convert the 
potential energy in this buoyant air layer to kinetic energy in an
anchored vortex, and to use that kinetic energy to drive a
vertical-axis turbine coupled with an electric generator in order to
produce electrical power. The mechanism is much like that of a naturally
%
% are we certain this is baroclinic? montegomery might argue it is not.
% update: I think montgomery is wrong here, or at best nitty.  
% BAROCLINIC: essentially just that temp fronts exist
%
occurring ``dust devil'' with baroclinic generation of vorticity in a
vertically stratified, ground-heated air layer producing a columnar vortex. 

With nearly one-third of global land mass covered by deserts, there are huge
untapped regions for capturing solar heat (about 200 W/$\text{m}^2$ averaged over
a 24-hour day, and up to 1000 W/$\text{m}^2$ peak).  The available power is
competitive in magnitude with worldwide power generation from fossil
sources. If successful, this could result in a low-cost, scalable
approach to electrical power generation that could create a new class of
renewable energy ideally suited for arid low-wind regions. 

The Solar-Driven Vortex (SoV) phenomena has already been demonstrated in
an experimental setup by our partners at Georgia Tech. The simulation
effort intends to utilize Computational Fluid Dynamics (CFD) to simulate
this SoV. These computer simulations are intended to discover the optimal
system configuration for a range of scenarios and system sizes. The
results of these simulations will be used as input for the design of a
pilot site in Mesa, Arizona, and eventually, over a range of
scenarios and system sizes. 

In order for these simulations to be generally useful, they must first
be validated against existing experimental data and high fidelity
simulations. These models will then explore regimes and scales where no
experimental measurements presently exist. 
%Characterizing the
%uncertainty of predictions resulting from extrapolation is a critical
%component in enabling reliable assessments of field performance of the
%SoV, as it will guide the commercialization strategy of the product. 


In addition to the system configuration, it is important to consider the
effect of local conditions on SoV performance. Characterizing the impact
of variations in ambient conditions on the SoV will guide the
commercialization strategy of the product, by determining optimal
install locations across the country. It is therefore desirable to have models
that are capable of accounting for variation in field conditions, such as solar
input, cross-winds and topography. Furthermore, it is expected that 
large ``farms'' of SoVs (akin to the wind and solar farms for wind
turbines and photovoltaics, respectively) may be used by commercial or
utility-scale energy generation. In order for this to be effective, 
the inter-unit spacing must also be optimized, as a single SoV collects
from a large area. These computations will guide commercialization
planning, where decision-makers will need to assess optimum unit size,
spacing, and geographic location for utility-scale deployment.  

This proposal is organized as follows. 
In section \ref{sec:method},
the algorithms used 

% provides a ``validation roadmap'' detailing the process by
%which an analysis of the uncertainties inherent to this
%problem may be characterized.  

It will begin with a discussion of the
physics scenario and the mathematical model, as well as detailing the
reliability of each submodel and the systems inputs. We will then
discuss the validation of these models against existing experimental
data and high fidelity simulations. 

\begin{itemize}
\item Motivation vis a vi energy and physics
\item ubiquitious
\item share a common structure
\item harnessx
\end{itemize}



This proposal is organized as follows. 
