\documentclass{article}
\usepackage{amsmath,amssymb}
\usepackage{hyperref}
\title{\bf{Laboratory Project \#1: Calibration of a Pressure Transducer}}
\author{Nicholas Malaya \\ Department of Mechanical Engineering \\ University of Texas at Austin} \date{}

\begin{document}
\maketitle
\date{}

This document details the calibration and uncertainty quantification of
an Omega PX164-005D5V pressure transducer. 

\section{Presentation of Calibration Data}

\textbf{You will have calibration data from using two different
standards and at least two repeats in each case.  
Present these results appropriately. Estimate the linear curve fit from
these data and display on the figures with the data.  
Ascertain whether a linear curve fit is appropriate or whether better
accuracy could be obtained with a different curve fit. Comment about
these results, particularly on any unusual or unexpected results.}  


[magus code] python read incline stats.py 
r-squared: 0.999552862197
p\_value 5.50240371955e-13

[magus code] python read micro stats.py 
r-squared: 0.990071511099
p\_value 2.84909524211e-08

\newpage
\section{Uncertainty Analysis}

\textbf{Estimate the uncertainty of the pressure measurements with the
pressure transducer using the linear curve fit you have determined in
your calibration. Compare this with the manufacturer's specifications of
an accuracy. Also estimate the lowest pressure differential that you can
measure with at least $\pm 10\%$ accuracy. Be sure to justify these
estimates with analyses.}

\end{document}
