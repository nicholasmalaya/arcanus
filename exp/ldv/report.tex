\documentclass{article}
\usepackage{amsmath,amssymb}
\usepackage{hyperref}
\usepackage{graphicx}
\usepackage{todonotes}

\title{\bf{Laboratory Project Two: Calibration of an Orifice Meter}}
\author{Nicholas Malaya \\ Department of Mechanical Engineering \\
University of Texas at Austin} \date{} 

\begin{document}
\maketitle
\date{}
\newpage
\section{Objectives}

\textbf{A short paragraph listing the specific objectives of this laboratory.}   

I believe the purpose of this laboratory was to learn about
several of the considerations necessary to properly operate an
LDV. Unlike the previous labs, the LDV had quite a bit of operational
details required to even begin gathering data. Our group took over an
hour to first start up the laser, play around with the oscilloscope,
etc. 

Beyond a familiarity with the LDV apparatus, this is also the first lab
where we have attempted to gather spatially resolved measurements. In
particular, we have characterized the free-stream turbulence (which we
expect to be statistically spatially independent) as well as the flow in
the wake of a cylinder. 

Common with the previous efforts, this work was also focused on
estimating the uncertainties (and ultimately, plausibility ) of our
measurements, both through uncertainty estimates as well as comparing
with available results in the literature. 

It was also my favorite lab. Felt very high tech. 

\section{Background and Experimental Details}

\textbf{Describe the basic operation of an LDV.}

I see the LDV as a four step process. 

\begin{itemize}
 \item Laser Crossing
 \item Particle movement through the fringes
 \item Light collected
 \item Signal postprocessing
\end{itemize}

Two laser beams cross, creating an interference pattern. Particles,
moving through this pattern reflect light in a similar oscillating
pattern, (some of) which is reflected back and collected. After signal
processing, the frequency of the signal of reflected light can be used
to calculate the quantity of interest, the particle's velocity. 

\textbf{Explain what frequency
shifting is, and why it is used. Calculate the following: 
\begin{itemize}
 \item Size of the Probe Volume
 \item Expected frequency response of the glass sphere seed particles
\end{itemize}
}   

We use frequency shifting so that we can measure flow
reversals. The frequency of light cannot be negative. 
If we did not shift the frequency of light scattered off
particles, we would have instances where particles moving backward
appeared to have the same frequency as particles moving
forward. Instead, by shifting the frequency forward, we can measure
frequencies that are lower than the effective frequency shift, which
permits us to determine the sign of the velocity.

The size of the probe volume is given by Stavros on page 267, 
\begin{equation}
 V_p = \frac{\pi d_{fe}^3}{6\text{Cos}(\theta/2)\text{Sin}(\theta/2)}
\end{equation}

However, we also need $\theta$, the intersection angle, here. 


% the doppler frequency of the particle is the normal velocity over the
% fringe spacing, which was: X-direction: 3.744um; y-direction: 3.551um
% (taken from Flowsizer) 

  % \begin{figure}[!htb]
  %  \begin{center}
  %   \includegraphics[width = 12 cm]{figs/Q_dP_fits.jpg}
  %   \caption{A plot of repeatability of the experiment. }
  %   \label{orif-zoom}
  %  \end{center}
  % \end{figure}

\section{Experiment One: Freestream}

\textbf{Indicate the frequency shift and bandpass filter settings used,
and why.} 

\section{Experiment Two: Wake Measurements}

\textbf{Present results from your wake measurements, including
uncertainty of the measurements. Compare your results with data in the
literature and comment on similarities and differences. } 


\section{Conclusions}


\textbf{Conclude by giving your opinion about whether your measurements
are more or less reliable than measurements in the literature.} 

%
%
%
\end{document}

% LocalWords:  reynolds H20 piecewise RDG yar
